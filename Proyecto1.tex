\documentclass[]{article}

%opening
\title{Proyecto 1: Métodos numéricos del Z-spline cúbico}
\author{}

\begin{document}

\maketitle

\begin{abstract}

\end{abstract}

\section{Los Z-splines}



\subsection{Introducción}

Los Z-splines constituyen una familia de splines cardinales de soporte
compacto que preservan momentos discretos y poseen alto orden de
regularidad. La idea central es construir una base spline 
cuya definición incorpora explícitamente operadores de diferencias 
finitas derivados de desarrollos de Taylor. Su creador, 
Julián T. Becerra-Sagredo, desarrolló esta teoría en dos trabajos clave
(2003 y 2020), en los cuales se presentan tanto la construcción como 
sus propiedades analíticas y aplicaciones.

\subsection{Definición cardinal y construcción básica}

Sea $\{x_j\} = \{j\}$ una malla equiespaciada.  
Los Z-splines se definen mediante un núcleo cardinal $Z_m(x)$ tal que la
interpolación de un conjunto de datos $\{y_j\}$ está dada por

\[
f_m(x) = \sum_{j\in\mathbb{Z}} y_j \, Z_m(x - j).
\]

Cada función base $Z_m$ es un polinomio por tramos de grado $2m-1$ y con
regularidad $C^{m-1}$. El parámetro $m$ determina el orden del spline.

La construcción impone condiciones de interpolación entre nodos y
conservación de momentos discretos. Para cada intervalo
$[j,j+1]$, el polinomio está determinado por un conjunto de condiciones
de tipo Hermite–Birkhoff derivadas de operadores de diferencias finitas.

\subsection{Conservación de momentos}

Una de las propiedades fundamentales es que las Z-splines preservan los
primeros $2m-1$ momentos discretos. Sea

\[
M_k = \sum_j y_j j^k, \qquad k=0,\dots,2m-2,
\]

entonces la interpolante satisface

\[
\int_{-\infty}^{\infty} x^k f_m(x) \, dx = M_k,
\]

lo que asegura que la interpolación no altera estos invariantes
(discretos o físicos).

\subsection{Reproducción polinómica}

El artículo demuestra que las Z-splines reproducen exactamente
polinomios hasta grado $2m-2$:

\[
p(x) = x^n, \qquad n \le 2m-2 \quad \Rightarrow \quad 
f_m(x) = p(x).
\]

Esto implica un orden de aproximación

\[
\| f - f_m \| = \mathcal{O}(h^{\,2m-1}),
\]

cuando $f$ es suficientemente regular.

\subsection{Soporte compacto y regularidad}

Cada $Z_m$ tiene soporte compacto mínimo:

\[
Z_m(x) = 0 \qquad \text{si } |x| > m,
\]

y regularidad

\[
Z_m \in C^{m-1}.
\]

Esto permite su uso eficiente en contextos numéricos, especialmente en
métodos de remapeo o mallas adaptativas.

\subsection{Relación con diferencias finitas}

Una contribución teórica clave es que los coeficientes de los polinomios
que forman $Z_m$ coinciden con los coeficientes de operadores de
diferencias finitas de orden elevado. En particular, se demuestra que
para derivadas,

\[
f_m^{(k)}(x_j) = \sum_{\ell=-m}^m c^{(k)}_{\ell} \, y_{j+\ell},
\]

donde los $c^{(k)}_\ell$ son coeficientes de diferencias finitas
centradas obtenidos por expansión de Taylor.

\subsection{Límite hacia la función sinc}

A medida que $m \to \infty$, las Z-splines convergen hacia la función
ideal de reconstrucción cardinal:

\[
Z_m(x) \longrightarrow \operatorname{sinc}(x)
= \frac{\sin(\pi x)}{\pi x}.
\]

Esto establece un vínculo entre splines compactos y teoría de muestreo.

\subsection{Z-splines en mallas no uniformes}

Los artículos extienden la teoría a mallas arbitrarias
$\{x_j\}$ no equidistantes. En ese caso, el núcleo cardinal ya no es
trasladable, y se construyen polinomios locales mediante matrices de
diferencias finitas adaptadas a la geometría de los nodos.

En regiones de frontera se emplean splines ``one-sided'', lo cual permite
resolver la falta de simetría.

\subsection{Aplicaciones numéricas}

Las Z-splines son especialmente útiles en contextos donde es esencial la
conservación de cantidades físicas discretas:

\begin{itemize}
	\item métodos de remapeo ALE y semi-Lagrangianos,
	\item interpolaciones conservativas en dinámica de fluidos,
	\item regularización de funciones delta,
	\item remallado adaptativo,
	\item filtrado numérico (denoising) preservando momentos.
\end{itemize}

En estas aplicaciones, la preservación de momentos juega un papel
fundamental para mantener estabilidad física del sistema.

\subsection{Conclusión}

Los Z-splines constituyen una herramienta moderna de
interpolación de gran interés tanto teórico como práctico, especialmente
en contextos de simulación numérica donde la preservación de invariantes
es esencial.


\section{El Z-spline cúbico} 

La función base Z-spline cúbica para datos a intervalos arbitrarios $a_{1}, a_{2}, a_{3}$ and $a_{4}$
está dada por 
\begin{displaymath}
\widetilde{Z}_{1} (x) = \left\{
\begin{array}{lll}
0&   $ $  &  x < -a_{1}-a_{2} \\ [0.2cm]
\left( \frac{a_{2}+a_{1}}{a_{1}} \right) + \left(
\frac{3a_{2}+a_{1}}{a_{2}a_{1}} \right) x
\\ [0.1cm] \quad  + \frac{3a_{2}+2a_{1}}{a_{2}a_{1}(a_{2}+a_{1})}
x^2 + \frac{1}{a_{2}a_{1}(a_{2}+a_{1})}x^{3} &
$ $  &  -a_{1}-a_{2} \leq x \leq -a_{2}, \\ [0.2cm]
1- \left( \frac{1}{a_{3}}-\frac{1}{a_{2}} \right) x \\ [0.1cm] \quad
- \frac{a_{3}+2(a_{2}+a_{1})}{a_{3}a_{2}(a_{2}+a_{1})} x^2
- \frac{a_{3}+a_{2}+a_{1}}{a_{3}a_{2}^2(a_{2}+a_{1})} x^3 &  
$ $  &  -a_{2} \leq x \leq 0, \\ [0.2cm]
1+ \left( \frac{1}{a_{2}}-\frac{1}{a_{3}} \right) x \\ [0.1cm]  \quad
- \frac{a_{2}+2(a_{3}+a_{4})}{a_{2}a_{3}(a_{3}+a_{4})} x^2
+ \frac{a_{2}+a_{3}+a_{4}}{a_{2}a_{3}^2(a_{3}+a_{4})} x^3 &  
$ $  &  0 \leq x \leq a_{3}, \\ [0.2cm]
\left( \frac{a_{3}+a_{4}}{a_{4}} \right) - \left(
\frac{3a_{3}+a_{4}}{a_{3}a_{4}} \right) x
\\ [0.1cm]  \quad  + \frac{3a_{3}+2a_{4}}{a_{3}a_{4}(a_{3}+a_{4})} x^2
- \frac{1}{a_{3}a_{4}(a_{3}+a_{4})}x^{3} &  
$ $  &  a_{3} \leq x \leq a_{3}+a_{4}, \\ [0.2cm]
0&   $ $ & x > a_{3}+a_{4}.
\end{array}
\label{eq:aiz1}
\right.
\end{displaymath}


\subsection{Z-spline cúbico en los extremos de un intervalo}

\section{Fórmula de integración numérica}

\section{Fórmula de derivación numérica}
\section{Convergencia}
\subsection{Integración}
\subsection{Derivación}
\subsection{Interpolaicón equidistante en 2D}
\subsection{Interpolación en 1D con puntos arbitrarios}


\end{document}
