\documentclass[]{article}

%opening
\title{Proyecto 1: Métodos numéricos del Z-spline cúbico}
\author{}

\begin{document}

\maketitle

\begin{abstract}

\end{abstract}

\section{Los Z-splines}


\section{El Z-spline cúbico} 

La función base Z-spline cúbica para datos a intervalos arbitrarios $a_{1}, a_{2}, a_{3}$ and $a_{4}$
está dada por 
\begin{displaymath}
\widetilde{Z}_{1} (x) = \left\{
\begin{array}{lll}
0&   $ $  &  x < -a_{1}-a_{2} \\ [0.2cm]
\left( \frac{a_{2}+a_{1}}{a_{1}} \right) + \left(
\frac{3a_{2}+a_{1}}{a_{2}a_{1}} \right) x
\\ [0.1cm] \quad  + \frac{3a_{2}+2a_{1}}{a_{2}a_{1}(a_{2}+a_{1})}
x^2 + \frac{1}{a_{2}a_{1}(a_{2}+a_{1})}x^{3} &
$ $  &  -a_{1}-a_{2} \leq x \leq -a_{2}, \\ [0.2cm]
1- \left( \frac{1}{a_{3}}-\frac{1}{a_{2}} \right) x \\ [0.1cm] \quad
- \frac{a_{3}+2(a_{2}+a_{1})}{a_{3}a_{2}(a_{2}+a_{1})} x^2
- \frac{a_{3}+a_{2}+a_{1}}{a_{3}a_{2}^2(a_{2}+a_{1})} x^3 &  
$ $  &  -a_{2} \leq x \leq 0, \\ [0.2cm]
1+ \left( \frac{1}{a_{2}}-\frac{1}{a_{3}} \right) x \\ [0.1cm]  \quad
- \frac{a_{2}+2(a_{3}+a_{4})}{a_{2}a_{3}(a_{3}+a_{4})} x^2
+ \frac{a_{2}+a_{3}+a_{4}}{a_{2}a_{3}^2(a_{3}+a_{4})} x^3 &  
$ $  &  0 \leq x \leq a_{3}, \\ [0.2cm]
\left( \frac{a_{3}+a_{4}}{a_{4}} \right) - \left(
\frac{3a_{3}+a_{4}}{a_{3}a_{4}} \right) x
\\ [0.1cm]  \quad  + \frac{3a_{3}+2a_{4}}{a_{3}a_{4}(a_{3}+a_{4})} x^2
- \frac{1}{a_{3}a_{4}(a_{3}+a_{4})}x^{3} &  
$ $  &  a_{3} \leq x \leq a_{3}+a_{4}, \\ [0.2cm]
0&   $ $ & x > a_{3}+a_{4}.
\end{array}
\label{eq:aiz1}
\right.
\end{displaymath}


\subsection{Z-spline cúbico en los extremos de un intervalo}

\section{Fórmula de integración numérica}

\section{Fórmula de derivación numérica}
\section{Convergencia}
\subsection{Integración}
\subsection{Derivación}
\subsection{Interpolaicón equidistante en 2D}
\subsection{Interpolación en 1D con puntos arbitrarios}


\end{document}
